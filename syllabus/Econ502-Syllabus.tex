%% ----------------------------
% Setup
%% ----------------------------
\DocumentMetadata{
  pdfstandard=UA-2,
  pdfversion=2.0,
  lang=en-US
}
\documentclass{syllabus}

\newcommand{\coursename}{ECON 502: Advanced Microeconomics}

%% ----------------------------
% Document information
%% ----------------------------

\hypersetup{
  pdftitle={\coursename},
}

%% ----------------------------
% Header
%% ----------------------------
\usepackage{fancyhdr}
\pagestyle{fancyplain}
\fancyhf{}
\lhead{ \fancyplain{}{\coursename} }
\rhead{ \fancyplain{}{Syllabus} }
\fancyfoot[c] {\thepage\ }
\thispagestyle{plain}


%% ----------------------------
% Begin document
%% ----------------------------
% Begin document content
\begin{document}
\courseheader{\coursename}{Spring 2026}{Department of Economics}{cbe-logo.png}

% Section 1: Faculty Information

\begin{center}
\begin{minipage}[t]{\textwidth}
\rule{\textwidth}{0.5pt} \\
\setstretch{1.15} 
\textbf{Instruction modality:} In-person \\
\textbf{Class and exam location:} SGMH 2501 \\
\textbf{Class days and time:} Mondays, 7-9.45 PM  \\ \vspace{-0.5em}

\textbf{Instructor:} Div Bhagia \vspace{-0.5em}
\begin{itemize}
\item[] \textbf{Office:} SGMH 3361 
\item[] \textbf{Email:} dbhagia@fullerton.edu
\item[] \textbf{Phone:} (657) 278-2914 
\item[] \textbf{Office hours:} Mondays, 5-6.45 PM, or by appointment (in-person or on \href{https://fullerton.zoom.us/j/81895171931}{Zoom})
\end{itemize}
%\hspace{3em} \textbf{Office:} SGMH 3361 \\
%\hspace{3em} \textbf{Email:} dbhagia@fullerton.edu, preferred (usually respond within 48 hours except on weekends) \\
%\hspace{3em} \textbf{Phone:} (657) 278-2914 \\
%\hspace{3em} \textbf{Office hours:} Mondays, 5-6.45 PM, or by appointment (in-person or on \href{https://fullerton.zoom.us/j/81895171931}{Zoom}) \\
\rule{\textwidth}{0.5pt} 
\end{minipage}
\end{center}


\section*{Technical Problems}
If you encounter any technical difficulties, contact the instructor immediately to document the problem. Then, contact: \href{https://www.fullerton.edu/it/services/helpdesk/}{student IT help desk}, \href{mailto:StudentITHelpDesk@fullerton.edu}{email}, phone = 657-278-8888, walk-in \href{http://www.fullerton.edu/it/students/sgc/index.php}{Student Genius Center}, online chat - log into \href{http://my.fullerton.edu}{Portal}; click ``Online IT Help''; click ``Live Chat.''

\noindent \underline{For issues with Canvas}: Canvas Support Hotline = 657-278-8888, \href{https://canvashelp.fullerton.edu/}{search the CSUF Canvas Guides with AI Assistant}, or \href{https://titans.service-now.com/sp?id=sc_cat_item&sys_id=f88efe80ebea6a10fb7cfcffcad0cdc6&subject=Canvas}{report a problem.}

\section*{Response Time}
I usually respond to emails or LMS messages within 48 hours (excluding weekends or holidays).

\section*{Course Communication}
All course announcements and individual emails are sent through CANVAS, which only uses CSUF email accounts. Therefore, you MUST check your CSUF email on a regular basis (several times a week) for the duration of the course.


\section*{Course Information}

\subsection*{Prerequisites}
ECON 310, ECON 441; Economics graduate standing.

\subsection*{University Catalog Description}

Advanced treatment of decision-making by individuals and firms. Optimal resource allocation in partial and general equilibrium contexts. Topics include choice and preferences, duality, utility maximization, profit maximization, risk and uncertainty, information economics, and market power.

\subsection*{Additional Course Description}

This course develops the core tools of microeconomic theory while connecting them to real-world applications and policy questions. 

We begin by examining how consumers decide what to buy (lectures 1-2) and how firms organize production (lecture 3), then consider how these behaviors determine prices and quantities in competitive markets (lecture 4). From there, we explore what it means for a market to work efficiently and how to think about fairness (lecture 5). We then turn to market power (lectures 6-8): What happens when a single or a few large firms dominate the market? How does market power affect wages in labor markets? 

In the second half of the course, we consider the need for government intervention when markets fail due to information asymmetries, externalities, or underprovision of public goods (lectures 9-10). Finally, we conclude by examining decision-making under uncertainty and strategic behavior through game theory (lectures 11-12).

\subsection*{Student Learning Outcomes}
Upon successful completion of this course, students will be able to:
\begin{enumerate}
\itemsep0em 
\item Model consumer and firm behavior and derive equilibrium prices and quantities in competitive markets
\item Apply welfare criteria to assess market efficiency and analyze tradeoffs between efficiency and equity
\item Analyze how market power affects prices, output, and wages in product and labor markets
\item Diagnose market failures arising from information asymmetries, externalities, and public goods, and evaluate policy responses
\item Apply expected utility theory to model choice under uncertainty and use game theory to solve strategic interactions
\end{enumerate}

\subsection*{Required Texts}

The following textbook is required for this course:

\begin{itemize}
\item \textit{Microeconomic Theory: Basic Principles \& Extensions by Walter Nicholson, Christopher Snyder, 13th Edition.}
\end{itemize}
Previous editions are also acceptable. Lectures will loosely follow the textbook. Supplementary readings, lecture slides, and problem sets will be posted on the course website.

\section*{Grading Policies and Standards}

\subsection*{Grading Scale}
In this course, the plus/minus system will be used. The grade breakdown is as follows:

\begin{itemize}
\item[] 98--100\% = A+ 
\item[] 93--97.99\% = A (outstanding performance) 
\item[] 90--92.99\% = A- 
\item[] 87--89.99\% = B+ 
\item[] 83--86.99\% = B (good performance) 
\item[] 80--82.99\% = B- 
\item[] 77--79.99\% = C+ 
\item[] 73--76.99\% = C (acceptable performance) 
\item[] 70--72.99\% = C- 
\item[] 67--69.99\% = D+ 
\item[] 63--66.99\% = D (poor performance) 
\item[] 60--62.99\% = D- 
\item[] 0--59.99\% = F 
\end{itemize}

\subsection*{Grade Breakdown}

\noindent Your grade will be determined according to the following breakdown: 
\begin{center}
\begin{tabularx}{0.65\textwidth}{Xc}
\hline
  \textbf{Assignment}        & \textbf{Points/Percentage} \\ \hline
Problem Sets & 10 \\
Midterm & 40 \\
Final Exam & 50 \\
\hline
Total & 100 \\
\hline 
\end{tabularx}
\end{center}

\subsection*{Problem Sets}

Four problem sets will be assigned throughout the semester. Each will be graded complete/incomplete and worth 2.5\%. To receive credit, you must submit a substantive attempt at all problems on time. Partial or blank submissions will receive no credit.

\subsection*{Examinations}

Exams will be administered in-person on the days indicated in the class schedule. These dates are fixed and will not be changed. Exams will include multiple-choice, short-answer, and long-form reflective questions designed to assess your understanding of the material and your ability to communicate these ideas fluently. All work on exams must reflect your individual effort. Students are expected to follow University policies for Academic Integrity.

Requests for re-evaluation of graded material must be made within one week
of the return the assignment. All requests must be accompanied by a
written explanation of your dispute. It is your responsibility to consult
with any posted rubrics or keys prior to making a re-grade request. Final exam will be \textit{cumulative}.

\subsection*{Extra credit}

There are no extra credit options in this course. 

\subsection*{Make-up and late submission policy}

Make-up exams will only be offered
under very limited circumstances. It is your responsibility to notify your
instructor either in advance or within 24 hours of missing an exam.
An assignment is considered late if it is posted/received past the due date
and/or time. You are encouraged to set personal deadlines ahead of required
due dates to allow for unforeseen events that prevent you from submitting
your work on time. Communicate immediately with your instructor if you
encounter a problem that may impact your ability to submit work on time.
Work may only receive a maximum of 50\% of the point total for the
assignments submitted after the original due date and time unless approval
for late work is given in advance. Specifically, late submissions will incur a 10\% deduction per day until the maximum late credit is reached.

%Late assignments are not accepted. Late assignments are defined as any
%assignment posted/received past the due date and/or time. All written
%assignments will be submitted through Canvas. You are encouraged not to
%procrastinate to avoid technological problems.
\subsection*{Generative AI Policy}

Thoughtful and transparent use of generative AI tools as learning aids is acceptable in this course. You may use AI tools like ChatGPT to help brainstorm ideas, clarify concepts, or improve your writing, but you must acknowledge their use and demonstrate your own critical thinking. All final work must reflect your own analysis and understanding. For major assignments, you must include a brief statement describing how you used AI tools, if at all. Using AI to complete assignments without engagement or to generate answers you cannot explain constitutes academic dishonesty. When in doubt about appropriate AI use, please consult with the instructor.

\subsection*{Alternative procedures for submitting work}

 In case of technical
difficulties with CANVAS or other online resources, your instructor will
communicate with students directly through their CSUF email. The instructor
will provide directions on alternative methods for submitting work and/or
may extend submission deadlines. If email is not available, students may
contact the department office for guidance.

\subsection*{Authentication of student work}

 All assignments submitted for a grade in this class must be your individual work
unless otherwise indicated. Student work will be authenticated by submission in class for in-person instruction and through Canvas for online instruction.

\subsection*{Retention of student work}

Work submitted for a grade in this course, either as a hardcopy or through the
CANVAS course site, shall be retained for a reasonable time after the
semester is completed not to exceed the last day of the subsequent
semester. Exam material is exempt from this policy; however, students have
the right to review their work in the presence of the faculty member. 

%%%%%%%% Academic Integrity
\section*{Academic Integrity}
It is expected that any work submitted for a grade in this course will be the sole
product of the individual student, unless otherwise permitted. Presenting
work from other sources as your own is unacceptable and will result in a
notification to the Dean of Students of a violation of campus standards.
Behaviors that seek to gain an unfair advantage or negatively impact the ability of others to learn will also result in a report to the Dean of
Students.

The consequences for cheating or other actions contrary to CSUF student
conduct polices can be severe. Communicate with your instructor if you find
yourself in a situation that might lead to a poor decision.

You are encouraged to use the resources provided by your instructor. Discuss
the use of other online resources with your instructor. The use of sites,
including but not limited to Chegg and Course Hero, which require
subscriptions and provide solutions to homework problems, exam questions,
etc., is explicitly prohibited in this course and is considered academic
dishonesty.

Collaborating has been made easier by the many tools available to use on the
internet (e.g. Discord, Zoom, Microsoft Teams). I encourage you to use these
tools to work together, to form study groups, etc. However, any sharing
of assignments (even if only intended to help) or using
these communication tools for unauthorized collaboration is considered
academic dishonesty. Unless otherwise explicitly stated by the instructor,
assignments, and examinations must be completed on your own.

\section*{Technical Requirements}
Students are expected to
\begin{enumerate}
\item Have basic computer competency which includes:
\begin{enumerate}
\item	The ability to use a personal computer to locate, create, move, copy, delete, name, rename, and save files and folders on hard drives, secondary storage devices such as USB drives, and cloud such as Google Drive (Titan Apps) and Dropbox
\item	The ability to use a word processing program to create, edit, format, store, retrieve, and print documents
\item	The ability to use their CSUF email accounts to receive, create, edit, print, save, and send an e-mail message with and without an attached file
\item	The ability to use an Internet browser such as Chrome, Safari, Firefox, or Microsoft® Edge to search and access web sites in the World Wide Web
\end{enumerate}

\item	Have ongoing reliable access to a computer with Internet connectivity for regular course assignments
\item	Utilize updated version of Microsoft® Office (for P.C. or Mac) including Word, PowerPoint, and Excel to learn content and communicate with colleagues and faculty; have the ability to regularly print assignments
\item	Maintain and access their CSUF student email account at least three times weekly
\item	Use Internet search and retrieval skills to complete assignments
\item	Apply their educational technology skills to complete expected competencies
\item	Utilize other software applications as course requirements dictate
\item	Utilize the LMS to access course materials and complete assignments
\end{enumerate}



\section*{CBE Assessment Statement}
The programs offered in the College of Business and Economics (CBE) at Cal State Fullerton are designed to provide every student with the knowledge and skills essential for a successful career in business. Since assessment plays a vital role in the college's drive to offer the best, several assessment tools are implemented to constantly evaluate our program as well as our students' progress. Students, faculty, and staff should expect to participate in CBE assessment activities. In doing so, the college can measure its strengths and weaknesses and continue cultivating a climate of excellence in its students and programs.

Assurance of Learning (AoL) is an integral part of both our AACSB and WASC accreditation. Please visit the \href{https://business.fullerton.edu/assessment}{Assessment and Instructional Support website} for more information on our college-based assurance of learning efforts, please visit the Assessment and Instructional Support website.

\section*{College and University Learning Resources}


\subsection*{Wall Street Journal}
You get a FREE subscription to the Wall Street Journal through the University. Activate your school-sponsored membership here: \href{https://partner.wsj.com/p/1110800011/register?mod=wsj_CSUF_360}{WSJ.com/ActivateCSUF}.


\subsection*{University Learning Center}
The goal of the University Learning Center is to provide all CSUF students with academic support in an inviting and contemporary environment.  The staff of the University Learning Center will assist students with their academic assignments, general study skills, and computer user needs. The ULC staff work with all students from diverse backgrounds in most undergraduate general education courses, including those in science and math, humanities and social sciences, as well as other subjects. They offer one-to-one peer tutoring, online writing review, and many more services.  More information can be found on the University Learning Center website. 

\subsection*{Writing Center}
The Writing Center offers 30-minute, one-on-one peer tutoring sessions and workshops aimed at providing assistance for all written assignments and student writing concerns. Writing Center services are available to students from all disciplines. Registration and appointment schedules are available at the Writing Center Appointment Scheduling System. Walk-in appointments are also available on a first-come, first-served basis to students who have registered online. More information can be found at the  Writing Center webpage.  The Writing Center is located on the first floor of the Pollak Library. Their phone number is (657) 278-3650.


% Section 9: Student Resources Website
\section*{Student Resources Website}

\noindent It is the student's responsibility to read and understand the required and important \href{https://fdc.fullerton.edu/teaching/student-info-syllabi.html}{student information for course syllabi}. Included is information about:

\begin{itemize}
\item University learning goals
\item General Education learning objectives
\item Netiquette/appropriate online behavior
\item Students' rights to accommodations
\item Campus student support resources
\item Academic integrity
\item Emergency preparedness/what to do
\item Library services
\item Student IT services and competencies
\item Software privacy and accessibility
\item Accessibility statement
\item Diversity statement
\item Land acknowledgement
\item Final exam schedule
\item Semester calendar
\end{itemize}


%%%%% Schedule
%\newgeometry{top=0.5in, bottom=1in, left=0.5in, right=0.5in} 
\section*{\centering Tentative Course Schedule} \vspace{1em}
{\renewcommand{\arraystretch}{1.15}
\begin{center}
\begin{tabularx}{1\textwidth}{|C{1.15cm}|C{1.35cm}|X|p{3cm}|C{1.25cm}|}
\Xhline{1.75\arrayrulewidth}
Date & Lecture  & Topics  & References & Due  \\
\Xhline{1.75\arrayrulewidth}
01/19 & \multicolumn{3}{c}{MLK Day} &  \\\Xhline{1.75\arrayrulewidth} 
01/26 & 1 & Consumer Preferences and Choice & Ch. 3-4 &  \\\Xhline{1.75\arrayrulewidth} 
02/02 & 2 & Demand Analysis and Consumer Welfare & Ch. 5-6 &  \\\Xhline{1.75\arrayrulewidth} 
02/09 & 3 & Production, Costs, and Firm Supply & Ch. 9-11 & PS 1 \\\Xhline{1.75\arrayrulewidth} 
02/16 & 4 & Competitive Market Equilibrium & Ch. 12 &  \\\Xhline{1.75\arrayrulewidth} 
02/23 & 5 & Welfare Analysis and Efficiency & Ch. 13 &  \\\Xhline{1.75\arrayrulewidth} 
03/02 & 6 & Monopoly and Market Power & Ch. 14 & PS 2 \\\Xhline{1.75\arrayrulewidth} 
03/09 & 7 & Imperfect Competition and Oligopoly & Ch. 15 &  \\\Xhline{1.75\arrayrulewidth} 
03/16 & 8 & Labor Markets & Ch. 16 &  \\\Xhline{1.75\arrayrulewidth} 
03/23 & \multicolumn{3}{c}{Midterm Exam} &  \\\Xhline{1.75\arrayrulewidth} 
03/30 & \multicolumn{3}{c}{Spring Recess} &  \\\Xhline{1.75\arrayrulewidth} 
04/06 & 9 & Asymmetric Information & Ch. 18 & PS 3 \\\Xhline{1.75\arrayrulewidth} 
04/13 & 10 & Externalities and Public Goods & Ch. 19-20 &  \\\Xhline{1.75\arrayrulewidth} 
04/20 & 11 & Choice Under Uncertainty & Ch. 7 &  \\\Xhline{1.75\arrayrulewidth} 
04/27 & 12 & Introduction to Game Theory & Ch. 8 & PS 4 \\\Xhline{1.75\arrayrulewidth} 
05/04 & 13 & Add. Topics/Review &  &  \\\Xhline{1.75\arrayrulewidth} 
05/11 & \multicolumn{3}{c}{Final Exam} & 
 \\
\Xhline{1.75\arrayrulewidth}
\end{tabularx}
\end{center}
%\thispagestyle{plain}
%\restoregeometry 

\end{document}